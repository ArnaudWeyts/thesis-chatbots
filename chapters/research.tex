\chapter{Research question}

The point of this bachelor thesis is to conduct research into chatbots and how they are built. What actually goes into building a chatbot that feels intuitive but is also functional?

\section{Approach}

This thesis will start with some use cases for chatbots. When it would be a good idea to build a chatbot as a solution for a concrete problem.

Already existing chatbots will be presented and compared to paint a picture of what a professional chatbot currently looks like.
Some general research into chatbot best practices and worst practices will also be conducted based on these real life cases and studies.

Finally, different platforms and frameworks to create a fully-fledged chatbot will be compared by building one.

\section{Metrics}

The metrics for the comparison will be based on lots of different factors. Their general business model will be researched, followed up by their learning curve and quality of documentation. Finally source code readability and scalability will be compared.

\chapter{General chatbot design}

\section{Choosing a chatbot solution}

\subsection{Introduction}

Why choose a chatbot as a solution to a problem? To find the answer to that question there should be a concrete description of the problem and research should be conducted into how that problem is currently handled.

\subsection{Use case 1: Food delivery}

Food delivery chatbots are some of the most common chatbots out there. That's because a basic chatbot for ordering food is easy to make and maintain. Most of the time they don't require complex language recognition and they will go through the same steps every time someone wants to order something.

A great example of this is Domino's. They own one of the most popular facebook messenger chatbots even though the bot itself is really simple.

Taking a look at the perspective of a new imaginary pizza place in town. This pizza place has a set menu with set formulas. They want to innovate and make it possible to receive and process online orders but don't want to lose the familiarity of their brand.

This is a perfect case for a chatbot. If they already have a facebook page, they can easily integrate a chatbot into it and promote it to their current customers. All they need on top of that is an admin panel for the company to maintain their menu and process orders.

Building a chatbot this way also opens up several possibilities for expansion in the future. They can easily start tracking customer's habits and improve their suggestions for specific customers.

There are some downsides to this solution as well however. Creating a chatbot from scratch for this purpose would cost a lot of money. That's why there are already solutions like Chatobook~\cite{chatobook} popping up. This is an all-in-one solution that provides a restaurant with a messenger bot and an admin panel to manage promotion/reservations/menu. You can also find templates for these kinds of bots that integrate with Google Sheets/Excel.~\cite{chatbot-templates-pizza}

The rise of general delivery services like Ubereats~\cite{ubereats} and Deliveroo~\cite{deliveroo} also complicates this matter. If those companies manage to integrate full ordering services into their platforms using chatbots -which they will most likely succeed in at one point- that would simplify everything a lot. But that also comes with a cost. The imaginary pizza restaurant would lose their unique identity as the chatbot's identity would be replaced with Uber's or Deliveroo's identity.

\subsection{Use case 2: Banking industry}

\subsection{Use case 3: E-commerce}