\addcontentsline{toc}{chapter}{Conclusion}
\chapter*{Conclusion}

As chatbots and machine learning are getting more relevant this day and age, it's interesting for companies to try and experiment with this technology. There's many challenges that arise when choosing to create a chatbot.

First, a company should ask themselves if they really need a chatbot. Chatbots do not fit every single use case. It's also important to create a personality for the chatbot, this should represent the company spirit and keep the user interested in the conversation. The use of emoji or casual sentences is a nice touch in certain contexts. An initial conversational flow should be constructed to guide the user. Creating a chatbot is not a one-time task, chatbots should evolve and get better over time, as the needs of the user become clear thanks to testing.

As for the practical implementation of creating a chatbot, it's important to first perform some research on different platforms. This day and age there are lots of platforms to create chatbots that practice their own way of creating one. The difference is already very noticeable between the Microsoft Bot Framework and Google's Dialogflow, as demonstrated by this thesis.

The Bot Framework seems to put its focus on programming a bot and provides lots of detailed documentation and walkthroughs. The Bot Framework integrates with different services to make use of natural language processing or storage. What the documentation does seem to run short on is setting up a developer environment. There is no standard to code a bot and deploy it automatically to the Azure platform at the time of writing this thesis. Testing the bot is very easy using the emulator, which is actively maintained.

Dialogflow takes a very different approach to all this. The platform integrates natural language processing right away and bases everything else on their web interface. The interface serves as the main guide to developing a chatbot, and there is plenty of documentation available on how to use it properly. There is only some minimal documentation provided to implement a webhook with custom logic but this webhook is still very much tied to the platform. Actually coding a chatbot using the full API would be a very hard task as Google only provides a raw API-reference. Because of Dialogflow's strategy, it doesn't make much sense to provide a separate emulator application like the Bot Framework does. The test interface on the platform suffices. Testing a webhook is not as self-evident, as it can't be tested locally and has to be deployed to integrate into the platform.

In conclusion there is no real winner between these two platforms, they both have their own strategies that might be more beneficial for one use case than the other. Dialogflow is an easier platform to get started on without requiring any coding skills. But the learning curve quickly increases once a more advanced bot has to be created. The Bot Framework's learning curve seems to be more gradual leading up to advanced cases but requires coding skills from the get-go.