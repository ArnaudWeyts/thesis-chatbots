\chapter{Microsoft Bot Framework}

The BotBuilder SDK is an open-source solution developed by Microsoft to write a chatbot in once and make it possible to connect that same chatbot to multiple channels. Examples are Skype, Slack, Facebook Messenger, etc.

\section{Business Model}

The Microsoft Bot Framework can be used in several ways. The base platform can be self-hosted on a server of choice. But to access the extra services, like spell checking, or recognizing user intents using Microsoft's LUIS (Language Understanding), an Azure subscription plan is required.

Azure is Microsoft's Cloud service that can be tailored to the customer's need. The customer can decide what modules or extra features he wants to add and calculate the price.

\begin{figure}[ht]
	\centering
	\includegraphics[width=\textwidth]{microsoft-azure-calculator-screen}\label{fig:microsoft-azure-calculator-screen}
	\caption{Microsoft's Azure Pricing Calculator~\cite{azure-pricing-calculator}}
\end{figure}

\section{Technical implementation}

Microsoft provides the bot framework for 2 languages: Node.JS and C\#. If C\# is used however, development can only be done inside of a Windows environment, or the limited online code editor.

One solution would be to use Node.JS in a code editor of choice. Preferably Visual Studio Code, Microsoft's cross-platform code editor. This way everything can be tailored and configured to the developer's needs.

\subsection{Developer environment}

Setting up a developer environment can be configured from scratch to be as complicated and complex as needed by the company or developer. Node.JS is an open-source, cross-platform JavaScript environment that can be run on a server. And in the end that's what a chatbot is, a sever API that responds to requests (messages or events from the user).

It's important to note that JavaScript or also called ECMAScript is a language that is advancing very quickly and Node.JS cannot keep up with all the newest yearly features. And all of the documentation involving the Bot Framework is written in ES5. However it's still possible to use all of the newest features of ECMAScript if you compile your code down the ES5.

The documentation on setting up an environment like this is very limited. Microsoft does not provide any instructions or boilerplates on how to set up an efficient environment. This allows for more freedom but increases the learning curve for a developer who wants to start coding a bot using newer ECMAScript features or the option to hot reload his code.

\subsubsection{Project structure}

Structuring the project is straight forward. Just like most production JavaScript projects there is a source folder containing all of the actual source code. Developers can decide to divide the test files into a separate folder but generally it's recommended to keep the test files close to the source file they reference. 

The node\_modules folder is a dependency folder, this contains any dependencies needed to build the project. There are two types of dependencies: Firstly there are developer-dependencies, these are dependencies needed to build the project and participate in development. An example is eslint, these are a set of specific language rules configured for the project to follow. There are also regular dependencies, these are external libraries the project might use to do calculations, or wrappers for certain technologies. The Microsoft Bot Framework works using their botbuilder SDK as a dependency.

Next up are the config files, these are the workhorses of the developer's environment. There are 2 separate configuration files for the developers to build and test the bot as for the production server to build a functional output file. Webpack is a very powerful, well documented build tool and allows for easy customization. It is commonly used for front-end development but after some tweaking it can be used for Node.JS development as well. The development config starts a server that hot-reloads any changes made to the source files. This way it's easy for the developer to seamlessly check his changes instead of manually recompiling and restarting the server. The production config is very straightforward and simply compiles a minified and optimized output file into the build folder.

Further more there is support for environment variables using the dotenv dependency. This file is not versioned to git and will contain any API keys used in the project.

\begin{figure}[ht]
	\centering
	\includegraphics[width=0.33\textwidth]{projectstructure}\label{fig:projectstructure}
	\caption{Project structure}
\end{figure}

\subsection{Testing}

The bot can be tested locally or remotely on any platform using Microsoft's own Bot Emulator~\cite{microsoft-bot-emulator}, which is also open-source. This provides the developer with a live testing interface and information about everything the bot receives, including requests and the data that comes with them.

Speech Recognition is also supported for speech enabled bots. Sending system activities is another useful feature because it allows the developer to emulate user events like for example a user joining the conversation. Lastly, payment processing is supported as well to emulate a transaction.

\begin{figure}[ht]
	\centering
	\includegraphics[width=\textwidth]{microsoft-bot-emulator}\label{fig:microsoft-azure-calculator-screen}
	\caption{Microsoft's Bot Emulator~\cite{microsoft-bot-emulator}}
\end{figure}

\section{Building the bot}

